\documentclass[12pt,table]{article}

\usepackage[table]{xcolor}
% TEMPLATE DEFAULT PACKAGES
\usepackage{amssymb,amsmath,amsfonts,eurosym,geometry,ulem,graphicx,xcolor,setspace,sectsty,comment,natbib,pdflscape,array,adjustbox,threeparttable}

% ADDED PACKAGES FOR THIS MANUSCRIPT
\usepackage{palatino,newtxmath,multirow,titlesec,threeparttable,tabu,booktabs,titlesec,threeparttable,mathtools,bm,bbm,subcaption,pdflscape,tcolorbox,mathrsfs,float}
% endfloat,



%\usepackage{kbordermatrix}% http://www.hss.caltech.edu/~kcb/TeX/kbordermatrix.sty
%\usepackage{amsmath}% http://ctan.org/pkg/amsmath

\usepackage[colorinlistoftodos]{todonotes}

\usepackage{afterpage}
\usepackage[hyphens]{url}
\usepackage[margin=1cm]{caption}

\usepackage[draft]{hyperref}
\newcommand{\tim}{$\,\times\,$}
% FIGURES & TABLES CAPTION STYLING
\captionsetup[figure]{labelfont={bf},name={Figure},labelsep=period}
\captionsetup[table]{labelfont={bf},name={Table},labelsep=period}

% SECTION TITLE SETTINGS
\titlelabel{\thetitle.\enskip}
\titleformat*{\section}{\large\bfseries}
\titleformat*{\subsection}{\normalsize\bfseries}

% COLUMN TYPES
\newcolumntype{L}[1]{>{\raggedright\let\newline\\\arraybackslash\hspace{0pt}}m{#1}}
\newcolumntype{C}{>{\centering\arraybackslash}p{5.2em}}
\newcolumntype{D}{>{\centering\arraybackslash}p{5em}}
\newcolumntype{E}{>{\centering\arraybackslash}p{6em}}
\newcolumntype{F}{>{\centering\arraybackslash}p{4em}}

\newcolumntype{R}[1]{>{\raggedleft\let\newline\\\arraybackslash\hspace{0pt}}m{#1}}

\newcolumntype{J}{>{\raggedright\arraybackslash}p{5em}}
\newcolumntype{K}{>{\raggedleft\arraybackslash}p{5em}}
\newcolumntype{M}{>{\raggedleft\arraybackslash}p{6em}}

\renewcommand{\arraystretch}{1.2} 

\newcommand{\regtext}{
Standard errors are clustered at the small-area level (in parentheses).  The water service area is divided into 2,974 small-areas.
\textsuperscript{c} p$<$0.10,\textsuperscript{b} p$<$0.05,\textsuperscript{a} p$<$0.01 \,\,
}


% MARGINS AND SPACING
\normalem
\geometry{left=1.1in,right=1.1in,top=1.0in,bottom=1.0in}
\setlength{\parskip}{2.5pt}

% SPECIAL CELL 
\newcommand{\specialcell}[2][c]{%
	\begin{tabular}[#1]{@{}l@{}}#2\end{tabular}}

% NO INDENT ON FOOTNOTES
\usepackage[hang,flushmargin]{footmisc}

\begin{document}

\begin{titlepage} 
%\title{{Borrowing with Unpaid Bills}\thanks{}}
%\title{{Microcredit from Delaying Bill Payments}\thanks{}}
\title{Microcredit from Delaying Bill Payments}
\author{\\[3em]
  William Violette\thanks{Federal Trade Commission, Washington, DC. E-mail: william.j.violette@gmail.com   Any opinions and conclusions expressed herein are those of the author and do not necessarily represent the views of the Federal Trade Commission or its Commissioners.  Many thanks to Matthew Panhans, Miriam Larson-Koester, Adrian Rubli-Ornelas, and Stefano Polloni.} \\
 \\ 
  }
\vspace{30mm}
\date{\vspace{5mm}This Version: \today}
\maketitle
\begin{abstract}

Delaying bill payments to public utilities may provide an important strategy for households with volatile incomes to smooth their consumption.  At the same time, tolerating late payments may reduce net revenues for utilities, which often leads to higher prices to cover costs.  Using billing records from a large water utility in Manila, this paper estimates a household consumption and savings model to evaluate counterfactual payment policies.  A popular proposal to ensure upfront payments --- prepaid metering --- recoups less revenue than is needed to compensate households for their loss of consumption smoothing.  Alternatively, a revenue-neutral policy allowing more late payments increases welfare by encouraging greater consumption smoothing.



\vspace{1in}
\textbf{Keywords:} credit constraints; consumption smoothing; water utilities. \\
\textbf{JEL Codes:} O13; E21; L95. \\
\bigskip
\end{abstract}
\setcounter{page}{0}
\thispagestyle{empty}
\end{titlepage}
\pagebreak \newpage

%\spacing{1.5}
\onehalfspacing

\section{Introduction}



Key notes to fill in later:
- Ignore externalities of booster pumps
- No quantity margin because of splitting of taps and because census data indicates really high coverage
- Reweight by household number
- Also measure by closest pipe for robustness
- Measure cost savings from less NRW! 
	-- flag potential spillovers in NRW that we're missing...
- could do non-linear pricing robustness check



% This paper focuses on the costs of imperfect maintenance. Most of the large literature on water and
% health has focused on initial access, not disruptions in water supply, and much of the work on water in
% the developing world has focused either on point of use solutions or on rural areas (see, e.g. Ashraf, Berry
% and Shapiro, 2010, Kremer et al, 2011). Fewtrell et al. (2005) and Esray et al. (1991) both provide
% meta-analyses of the public health literatures on water and health. Typically, the papers that measure the
% impact of piped water infrastructure find a negative correlation between piped water and disease (Merrick,
% 1985; Galiani, Gertler and Schardrogsky, 2005; Gamper-Rabindran, Khan and Timmings, 2010), but not
% in all cases (Fewtrell et al., 2005), perhaps because of maintenance problems, or perhaps because better
% water reduces the incentives to engage in sanitary behavior (Bennett, 2012). Devoto et al. (2012) find
% that increased household connections in urban areas lead to no difference in health but an improvement in
% subjective well-being. Sasaki et al. (2008) also examine Lusaka, and test the impact of rainfall and poor
% drainage on cholera epidemics in Lusaka.




\section{Data and Setting}

As a pioneer in water infrastructure investments among developing cities, Manila provides a useful context to study the welfare effects of these types of investments.  In 1997, Manila awarded private concession contracts to two private companies to take over for the existing public utility.  Each company is assigned to provide water to their assigned halves of Metro Manila.  Both companies have gradually replaced the decaying pipe infrastructure inherited from the public utility.  

To measure the effect of pipe replacements, one of the companies has provided a map of their water pipes including the installation year for each pipe as well as water billing records for their 1.5 million water connections from January, 2008 to June, 2015.  To measure each water connection's exposure to pipe replacement projects, connections are first located within ``small areas'' --- the smallest geographic designations used by the company each containing around 270 connections.  Next, since projects often replace many pipes in the same place at the same time, each small area is assigned a ``pipe replacement year'' according to the year when the greatest total length of tertiary pipes were installed within that small area.\footnote{Tertiary pipes installed in the replacement year account for 85\% of the total tertiary pipe length in small areas.}  Tertiary pipes act as local feeders transporting water from large primary and secondary pipes directly to households.  Therefore, tertiary pipe replacements likely affect service quality within local areas while primary and secondary pipe replacements may have diffuse impacts throughout the pipe network.  By excluding primary and secondary pipe replacements, this approach limits the potential for spillover effects of pipe replacements onto neighboring areas at the cost of potentially underestimating the total welfare gains from these projects.  

Billing records measure monthly water prices and usage for each water connection, which may serve multiple households.  In order to link connection-level consumption to household welfare, billing records for each connection are merged to a survey of water connections conducted by an independent evaluator to monitor compliance with the water utility's service obligations.  This connection survey records demographics for the household that owns the connection as well as the number of other households and people that also use the connection.  Household-level water usage is calculated by dividing total connection usage by the number of households sharing the connection.  The survey also includes household demographics for the connection owner, different measures of water service quality, as well as household investments in booster pumps, water filters, and water storage tanks.

The survey randomly interviewed water connection owners covering 15,000 connections in 2008, 24,000 connections in 2010, and 23,000 connections in 2012.  Because a similar sampling design was followed across survey rounds, 13\% of connections were interviewed in two years, and 1.4\% of connections were interviewed in three years while remaining connections were interviewed in only one year.  Since billing records form a monthly-panel, survey responses must be interpolated

Since billing records are 


To examine household demographics, billing data are merged at the connection-level to a water connection survey conducted independently to monitor the quality of the utility's service.  




\footnote{Households using connections alone tend to be larger and wealthier than households that share connections with other households according to previous research (\cite{wjv}).}  

% The connection survey includes demographics for the households that own their water connections.  Table~\ref{table:sampleconstruction} in Appendix~\ref{appendix:sampleconstruction} includes more details on how the final sample is constructed for the analysis.
% This sample restriction ensures that connection-level is equal to household-level.


The water connection survey also includes measures 




Additional survey data also includes information on demographics and water quality as well as investments!

Water pipes are linked to areas... 
The utility locates water connections within small areas each containing around 270 connections.  

Each small-area is assigned a pipe replacement year as the year when the greatest total length of tertiary pipes are installed within the small area.   This pattern is consistent with 


  This calculation includes only tertiary pipes (excluding )

 which allows for linking 


The pipe map documents the installation year for each length of pipe, which forms the basis for the pipe replacement measure.   The 



% For the company, these efforts ramped up dramatically between 2008 and 2015.  
% Identifying the effects of water pipe investments requires detailed information on the condition and location of water pipes that are often kept confidential out of security concerns.  
% A regulated water utility serving half of the area of Metro Manila, Philippines has provided confidential access to a map of the water pipes as well as water billing records for 1.5 million water connections.  



% This utility provided access to monthly billing records for each connection as well as detailed information covering the regulatory structure and costs of production.  Monthly billing records include meter readings, billing amount, outstanding balances, and payments spanning January, 2008 to May, 2015. Over this period, the total number of connections increased from 900,000 to 1,500,000 as the water utility expanded service access.  Water connections are split into four categories: residential (90\%), semi-business (4\%), commercial (5\%), and industrial (1\%).  
% To focus on household consumption decisions, only residential connections are included in the analysis.



% A household panel survey from Pasay City provides information on household income dynamics in Manila.  Pasay City is a centrally located sub-municipality of Metro Manila accounting for 3.2\% of the population of Metro Manila as of 2015 and is roughly representative of households living in the utility's service area.  The panel survey covers over 60\% of the population of Pasay City and is the only survey to provide a household-level income panel over a similar time frame in Manila.  The analysis includes data from \input{tables/total_inc_hhs_all}households that were interviewed in both 2008 and 2011.



Section: Data

	- only households connected earlier (explain) (what share are those?!)
	- consumption per HH
	- reweight data for household level?! YES!
	- survey panel and other panel; date representation, how its imputed, etc.
	- Put in a table describing pipe-replacement;  describe staggered approach; predict pipe-replacement?
	- industrial/commercial footnote later!!

Consumption is measured according to average consumption for not-shared households!  What extra assumption is that!!!  NEED ROBUSTNESS!!! 

MEASURING AVERAGE PRICES


\section{Descriptives}

Descriptive evidence indicates that fixing pipes improves water pressure, quality, and reliability, which may each affect consumer welfare.  Table~\ref{table:descriptives} tracks water service improvements by comparing average household survey responses before and after pipe replacement.   [summarize the findings]

Consumers also invest in products and behaviors to compensate for low piped water service quality.  These investments help reveal which aspects of piped water service are most valuable to consumers as well as most affected by pipe replacement.  

Large investments in booster pumps suggest that households strongly value water pressure.  Booster pumps are both expensive to purchase and operate. In Manila, booster pumps range in price from 1,200 to 15,000 PhP, which represents a large expense given average monthly household incomes of \input{tables/median_inc}PhP.\footnote{Figures come from scraping results from the first page of searching ``booster pumps'' on the popular online marketplace in Manila, \url{https://www.lazada.com.ph/},  yielding 24 entries with horsepower listed.  Average monthly household income is computed for Metro Manila using the 2015 Family Income and Expenditure Survey.}  The average booster pump uses a 0.9 horsepower engine, which generates monthly costs of around 486 PhP at prevailing energy prices.\footnote{A 1 horsepower engine uses around 0.786 Kw per hour.  Table~\ref{table:descriptives} indicates that households use their booster pumps for 2.6 hrs per day, which implies around 78 hours per month.  In 2012, tariffs for the electricity utility in Manila averaged around 8.8 PhP per KwH.}   Before pipe replacement, 40\% of households invest in booster pumps that increase water pressure.  After pipe replacement, the share of households using booster pumps drops to 15\%.  This finding is consistent with booster pumps providing an important substitute to pressure from new water pipes.
% = .9*.786*78*8.8
% \footnote{A 1 horsepower engine uses around 0.786 Kw per hour.  Table~\ref{table:descriptives} indicates that households use their booster pumps for 2.6 hrs per day, which implies around 78 hours per month.  In 2012, tariffs for the electricity utility in Manila averaged around 8.8 PhP per KwH.}
Small investments in water filters combined with frequent purchases of filtered water both before and as well as after pipe replacement indicate that households may not derive large benefits from improvements in piped water quality.  Only 12\% of households use water filters both before and after pipe replacements.  Low filter usage may stem from the fact that less than half of households report drinking from the tap while over 70\% of households report purchasing filtered water from local water-refilling stations.  These behaviors remain constant after pipe replacements.  Taken together, these findings indicate that water quality improvements from pipe replacements may not be primary drivers of changes in household welfare.

Households cope with unreliable water supply by investing in water storage tanks.  Before pipe replacement, 43\% of households report using water storage tanks.  This percentage only drops to 36\% following pipe replacement, which is consistent with households continuing to report frequent water outages even after pipe replacement.  While households seems to value reliable service, small changes in storage tank use suggest that reliability improvements are unlikely to account for a large share of the welfare gains from pipe replacements.

\begin{table}[h!] 
\centering
\caption{Average Survey Responses Before and After Pipe Replacement}\label{table:descriptives}
\vspace{-2mm}
\begin{threeparttable}
\begin{tabular}{@{}l*{1}{KKK}@{}}
\toprule
  & Before & After  & All \\
\midrule
Piped Water Service Quality \\[.5em]
\hspace{1em}$\text{Water has strong pressure (6pm-12am)}^{\dagger}$ & 0.25 & 0.58 & 0.48 \\
\hspace{1em}$\text{Water has no pressure (6pm-12am)}^{\dagger}$  & 0.30 & 0.06 & 0.11 \\
% \hspace{1em}$\text{Hours with pressure per day}^{\dagger}$ & 21.46 & 23.44 & 22.48 \\
\hspace{1em}Water interruptions in last 3 months & 2.27 & 1.94 & 2.13 \\
\hspace{1em}Water has foreign bodies & 0.24 & 0.04 & 0.12 \\
\hspace{1em}Water is discolored & 0.09 & 0.04 & 0.05 \\
\hspace{1em}Water has unusual taste/smell & 0.12 & 0.03 & 0.05 \\
\\[-.5em]
Household Service Quality Investments \\[.5em]
\hspace{1em}Has booster pump & 0.37 & 0.11 & 0.16 \\
\hspace{1em}Hours booster pump is used per day & 2.84 & 2.47 & 2.50 \\
\hspace{1em}Has water storage tank & 0.43 & 0.36 & 0.39 \\
\hspace{1em}Has water filter & \input{tables/filter} \\
\hspace{1em}Purchases filtered water & 0.70 & 0.76 & 0.70 \\
\hspace{1em}Purchases from a deepwell & 0.05 & 0.02 & 0.03 \\
\hspace{1em}Spending on non-piped water (PhP) & 90.95 & 88.43 & 87.04 \\
\hspace{1em}Drinks from the tap & 0.47 & 0.43 & 0.49 \\
% \hspace{1em}Boils tap water before drinking & 0.23 & 0.19 & 0.21 \\
\\[-.5em]
Demographics \\[.5em]
\hspace{1em}Household size & 4.92 & 4.94 & 4.96 \\
\hspace{1em}Employed members & 1.62 & 1.49 & 1.57 \\
\hspace{1em}High-skilled employment & 0.11 & 0.08 & 0.09 \\
\hspace{1em}Lives in duplex& 0.17 & 0.20 & 0.18 \\
\hspace{1em}Lives in single house& 0.49 & 0.46 & 0.51 \\
\hspace{1em}Number of other households sharing tap& 0.84 & 0.86 & 0.90 \\
\\[-.5em]
Households & 10,235 & 9,136 & 49,319 \\
\bottomrule
\end{tabular}
\begin{tablenotes}
\footnotesize
\item  $\dagger$ when not using booster pump.  Bill, Unpaid Balance, Payment, and Income are in PhP.  Measures exclude months where households remain disconnected through the end of the sample period.  Billing data include households for household-month observations.  Income data include households for household-month observations.  45 PhP = 1 USD \,\,
% See Appendix~\ref{appendix:sampleconstruction} for more details on the sample construction.
\end{tablenotes}
\end{threeparttable}
\end{table}


Increases in water pressure from pipe replacements may lead households to increase their monthly water consumption.  Rapid water flow allows households to complete a greater number of water-using activities like cleaning and bathing in the same amount of time.  Greater water pressure may also allow households to engage in new activities that require minimum pressure levels such as showering.\footnote{By contrast, water quality improvements may have an ambiguous effect on water consumption.  On one hand, cleaner water may induce households to use more tap water for cooking and cleaning.  On the other hand, cleaner water may increase the productivity of cleaning and bathing, which may lead households to use less piped water.}  Figure~\ref{figure:pipecons} plots average monthly water usage per household in the 4 years before and 6 after pipe replacement.  Usage increases from an average of \input{tables/c_pre}m3 before pipe replacement to \input{tables/c_post}m3 after pipe replacement, which represents an \input{tables/c_diff_per}\unskip\% increase.  The increase in usage occurs abruptly at the year of pipe replacement and remains at roughly the same level in the following 6 years.  This sustained increase in consumption suggests that pipe replacement may provide sustained impacts on household welfare.  

The absence of strong pre-trends in usage suggests that replacement projects are not targeted to areas with particular trends in local water demand.  Instead, this pattern is consistent with the company's stated goal of sequentially replacing old pipes according to engineering specifications.  Table~\ref{table:descriptives} further supports this theory by indicating few demographic differences between households that receive pipe replacement projects (columns (1) and (2)) and all households (column (3)).  Also, demographic characteristics appear relatively similar before and after pipe replacement, which suggests that pipe replacements were not coupled with other policies that may have also affected demand for water.  

\begin{figure}
\begin{center}
\caption{Average Consumption per Household with Years to Pipe Replacement}\label{figure:pipecons}
\includegraphics[scale=1]{tables/pipe_cons.pdf}
\end{center}
%%% add counts and average change in discussion
\end{figure}
% discuss dynamics!!! of figure!!

Mapping these increases in consumption into household welfare requires measuring how households trade off water usage and price.  Prices are determined by the government regulator who imposes an increasing tariff schedule according to monthly usage, which is standard among public utilities (\cite{hoque2013state}).  Despite steep increases in marginal price at specific levels of usage, households do not appear to be sensitive to these price changes because households are not observed adjusting their consumption strategically to avoid higher prices (cite appendix).  The regulator also increases prices gradually over time to ensure that the company continues to cover operating costs.  Since households also gradually increase their average usage over this interval, it is unclear whether households are responsive to these price increases (cite appendix).

[ RE-WRITE WITH NEW DEFINITION! ]
The government regulator also gives the water company discretion to assign households to a high-price tariff schedule if they have any business activity at their residence, which provides a novel source of price variation in the context of public utilities.  In the vast majority of cases, the high price tariff is applied to households that operate small food stands (or ``Sari-Sari'' stores).  For a household with average monthly usage, the high-price tariff results in an average price of \input{tables/p_s}PhP/m3 while the regular tariff results in an average price of \input{tables/p_r}PhP/m3 (cite appendix for tariff).  The water company periodically visits consumers and updates prices according to the activities observed at each consumer's residence.  In some cases, consumers request price changes, which prompts the company to investigate the household and determine the appropriate price.  Table~\ref{table:pricechangestatistics} provides average characteristics of households that are always observed with the regular price (in column (1)), that are always observed with the high price (in column (2)), and that are observed changing prices during the sample (in column (3)).  While the 2,442 households that experience price changes use more water than other households, they share similar demographic characteristics, which suggests that they may also share similar price-sensitivities.  

Figure~\ref{figure:usagepricechanges} plots average usage 4 years before and after households are switched from the regular price to the high price (in red) as well as before and after households are switched from the high price to the regular price (in blue).  Before the price change, average usage for both groups follows relatively constant trends.  The price change is associated with a usage jump for households switched to the regular price and a corresponding usage slide for households switched to the high price.  The changes in consumption appear relatively persistent up to 4 four years after the price changes.  These patterns indicate that household water usage is sensitive to large, discrete price changes.  


\begin{table}[h!] % PUT IN DEMOGRAPHICS PLEASE ?!?!?
\centering
\caption{Average Household Characteristics by Prices Charged}\label{table:pricechangestatistics}
\vspace{-2mm}
\begin{threeparttable}
\begin{tabular}{@{}l*{1}{KKK}@{}}
\toprule
  & Always Reg. Price & Always High Price  & Change Reg. to High Price \\
\midrule
Usage per Household (m3)& 19.91 & 19.81 & 23.40 \\
Household size & 4.98 & 4.68 & 4.99 \\
Employed members & 1.56 & 1.45 & 1.50 \\
High-skilled employment & 0.15 & 0.08 & 0.10 \\
Lives in duplex& \input{tables/sub_rs} \\
Lives in single house& 0.50 & 0.59 & 0.54 \\
Other households sharing tap& 0.43 & 0.44 & 0.46 \\
\\[-.5em]
Households & 43,494 & 3,818 & 2,007 \\
\bottomrule
\end{tabular}
\begin{tablenotes}
\footnotesize
\item  Reg. refers to regular price.  

Pressure is 6 to midnight! Bill, Unpaid Balance, Payment, and Income are in PhP.  Measures exclude months where households remain disconnected through the end of the sample period.  Billing data include households for household-month observations.  Income data include households for household-month observations.  45 PhP = 1 USD \,\,
% See Appendix~\ref{appendix:sampleconstruction} for more details on the sample construction.
\end{tablenotes}
\end{threeparttable}
\end{table}

\begin{figure}
\begin{center}
\caption{Usage and Price Changes}\label{figure:usagepricechanges}
\includegraphics[scale=1]{tables/r_to_s_graph.pdf}
\end{center}
\end{figure}

\section{Model}

A simple model of household water demand connects changes in water usage and investments in water booster pumps to changes in household welfare as a result of pipe replacements.  In this model, households choose their monthly water usage as well as whether to use a booster pump to boost their water pressure.  

Each month, household utility takes the following form
\begin{align}
\label{eq:utility}
U\,=\,\frac{1}{\alpha} \, \Big[ \,  Q(B,R) \,  w  \, -\, \frac{1}{2}(w - \gamma)^2 \, \Big] \, + \, x 
\end{align}
where $w$ is water consumption and $x$ represents a bundle of all other goods consumed by the household.  Water service quality given by the function, $Q(B,R)$, depends on booster pump use where $B$ equals 1 if the household chooses to use a booster pump (and zero otherwise) and as well as pipe replacements where $R$ equals 1 after pipe replacement (and zero otherwise).  $Q(B,R)$ is assumed to be differentiable in $R$.  Water service quality enters multiplicatively with water consumption, which nests the assumption that each unit of water consumption is affected equally and positively by water service quality.  This assumption also excludes the possibility that improved water service quality affects household utility in ways that are not directly proportional to household water consumption such as by increasing housing values.  In this case, the approach would underestimate the welfare benefits of improved service quality.  $\alpha$ reflects price sensitivity, and $\gamma$ reflects the satiation amount of water usage.  This approach assumes that preferences for water are quasi-linear, which implies that water consumption does not depend on household income [ Footnote to heterogeneity table ].  


% $\epsilon$ captures idiosyncratic monthly variation in satiation water usage such as changes in weather or in the number of household members using water.  $\epsilon$ is assumed to follow a normal distribution with zero mean and standard deviation $\sigma_{\epsilon}$.  
% The $\theta$ terms capture the household's preference for water service quality from booster pumps, pipe replacements, and the combination of the two.  
% For example, $\theta_3<0$ implies that booster pumps and pipe replacements are substitutable in producing water service quality.  

Households maximize utility subject to the following budget constraint
\begin{align}
\label{eq:bc}
p\,w \,+\, x \, +  B \, F \, = Y
\end{align}
where $p$ is the average price of water while the price of all other goods, $x$, is normalized to 1.  This approach includes the simplifying assumption that households respond to a single, average water price $p$ despite facing marginal prices that increase with monthly consumption.  This assumption is consistent with descriptive evidence that households appear unresponsive to marginal prices (cite appendix).   $F$ is total cost of renting and using a booster pump each month.  This approach assumes that Manila has a competitive market for renting/selling booster pumps that is unaffected by improvements in piped water service quality.\footnote{This assumption is consistent with local service quality improvements and a city-wide market for booster pumps.}  $Y$ is monthly household income.  

% $\upsilon$ captures idiosyncratic monthly variation the booster pump cost such as changes in the electricity price or variation in installation costs. $\upsilon$ is assumed to follow a normal distribution that is assumed to be uncorrelated with $\epsilon$ and has zero mean and standard deviation $\sigma_{\upsilon}$.  

% as well as other research on public utility price responsiveness (\cite{ito2014consumers}, and the water ito paper?).  

Maximizing household utility in equation $(\ref{eq:utility})$ subject to the budget constraint in equation $(\ref{eq:bc})$ for a given choice of booster pump usage, $B$, yields the following expression for monthly water demand
\begin{align}
\label{eq:demand}
w^{*} \, = \, \gamma \, - \, \alpha p +  Q(B,R)
\end{align}
Water demand depends linearly on the satiation preference, $\gamma$, price, and service quality.

Given optimal water demand, the indirect utility function at the optimal booster pump choice, $B^{*}$, is given by 
\begin{align}
V\,=\,\frac{\alpha p^2}{2} - \gamma p + \frac{Q(B^{*},R)^{2}}{2\alpha} - p Q(B^{*},R) + \frac{\gamma Q(B^{*},R)}{\alpha} + y - B^{*} F
\end{align}
The derivative of $V$ with respect to pipe replacements, $R$, reflects the change in consumer welfare associated with pipe replacements.  This derivative takes the following form after substituting terms for $w^{*}$
\begin{align}
\label{eq:dvdr}
\frac{dV}{dR}\,=\,\frac{w^{*}}{\alpha} \frac{dQ}{dR} - F \frac{dB^{*}}{dR}
\end{align}
This expression summarizes the consumer welfare effects of pipe replacements in terms of (1) the marginal effect on service quality weighted by water consumption and price sensitivity as well as (2) the marginal effect on booster pump usage weighted by the cost of booster pumps.  This approach assumes that welfare effects are proportional to consumption, $w^{*}$, implying that service quality improvements affect all units of water consumption.  Welfare effects are decreasing in price sensitivity, $\alpha$, which captures the intuition that households with many substitute water sources (ie. high $\alpha$) may benefit less from improvements in service quality.  According to this approach, welfare changes do not require measuring fixed preferences, $\gamma$, separately from levels of service quality, $Q(B^{*},R)$.  Therefore, this approach is robust to different patterns of selection into booster pump use based on different levels of service quality or fixed preferences for water consumption.  

% This feature is especially useful in the empirical setting where booster pump use is non-random across households.  
% The advantage of this approach is that welfare estimates do not require 
% Advantage in terms of unobservables, disadvantage in terms of counterfactuals...
% - Surplus is increasing in consumption (by assumption! given the multiplicative aspect)


\section{Empirical Strategy}

The estimation strategy recovers household price sensitivity as well as the changes in service quality and booster pump use in response to pipe replacement projects. These estimated quantities along with observed water use and booster pump costs summarize the change in consumer welfare from pipe replacement projects as given by equation (\ref{eq:dvdr}).  The estimating equation takes the following form
\begin{align}
\label{eq:esteq}
y_{itl} \,=\, \beta_1 \, \text{After Pipe Replacement}_{tl} \,+\, \beta_2 \, p_{it} \, + \, \theta_t \, + \, \lambda_i \, + \epsilon_{itl}
\end{align}
where the outcome, $y_{itl}$, is either monthly water consumption or booster pump use.  Outcomes are measured for household, $i$, in small area, $l$, at calendar month, $t$. After Pipe $\text{Replacement}_{tl} $ takes a value of one for months after pipe replacement and zero otherwise.  Since the data identify only the year that pipes were replaced, all months from the replacement year onwards are considered after pipe replacement. $p_{it}$ measures the average price faced by each household in each month. $\theta_t$ includes calendar month fixed effects while $\lambda_i$ includes household fixed effects.

When the outcome is monthly water consumption, the estimating equation maps directly onto the equation for optimal household consumption given by equation ($\ref{eq:demand}$).  $\beta_1$ captures the effect of pipe replacement on service quality, $\frac{dQ}{dR}$.  Likewise, $\beta_2$ captures the negative of household price sensitivity, $\alpha$.  Remaining fixed preferences for water and levels of service quality are absorbed in household and month fixed effects, $\lambda_i$ and $\theta_t$.  These fixed effects ensure that the effect of pipe replacements is identified from variation in water consumption for the same household before and after pipes are fixed while accounting for any temporal consumption patterns common to all households.  Likewise, price sensitivity is estimated from variation in price for the same household switching between regular and high prices.  Calendar month fixed effects absorb price variation affecting all households equally over time as the company updates water tariffs.  

Identifying $\beta_1$ and $\beta_2$ requires assuming that pipe replacements and household-specific price changes are not correlated with any other factors that may also affect water demand at the same time.  One concern may be that pipe improvements are paired with other infrastructure investments that drive increases in local income, population, and water demand.  Alternatively, the water company may strategically target pipe improvements to either boost areas with declining demand or accelerate areas with growing demand.  Figure~\ref{figure:pipecons} traces a sharp break in average usage at the year of pipe replacement with little evidence of strong trends before or after replacement projects.  These patterns are consistent with company reports that pipe replacements were not explicitly linked to other infrastructure projects and were instead planned primarily to minimize engineering costs.

Another concern is that households may increase their water demand as they are switched to the high price because they often open roadside food-stands.  Conversely, households that close their roadside food-stands may decrease their consumption at the same time as they are switched to the low price.  Both cases suggest that the empirical approach would underestimate household price sensitivities.  Figure~\ref{figure:usagepricechanges} provides little evidence of strong pre-trends leading up to price changes.  While this evidence is suggestive that these events are uncorrelated with long-term trends in water demand, this approach is unable to exclude the possibility that other factors may influence water demand at the same time as these events.

When the outcome is booster pump use, the estimating equation captures the effect of pipe replacements on booster pump use.  This approach assumes that demand for booster pumps can be linearly approximated by the estimating equation such that $\beta_1$ reflects $\frac{dB}{dR}$.  Since booster pump use is a binary outcome, this approach represents a linear probability model.  Time fixed effects account for variation in booster pump usage over time common to all households while household fixed effects account for differences in booster pump use across households.  $\beta_1$ reflects the effect of pipe replacements on booster pump use under the assumption that no other factors independently affect booster pump use at the same time.  One concern may be that the prices for booster pumps change in response to pipe replacement projects.  Pipe replacements occur in small geographic areas while markets for booster pumps are likely to cover much wider areas.\footnote{Online markets span the entire city.}  Therefore, individual pipe replacement projects are unlikely to influence prices for booster pumps.

% This approach also assumes that booster pumps do not impose substantial negative externalities on local water service quality.  A recent engineering literature has raised the possibility that booster pumps interfere with water flow in pipe mainlines, reducing pressure along the entire pipe (\cite{taylor2014reducing}).  However, this problem may not be as salient in the context of Manila.  The water company has not raised booster pumps as a policy problem in any of their internal documents.\footnote{Booster pumps are sometimes mentioned in water company investigations of consumer complaints as an explanation for household water pressure and never as problem to be addressed.}  Moreover, externalities would predict large water usage gaps between those with and without booster pumps in the same neighborhood.  By contrast, Table~\ref{table:usageregs} finds a similar association of booster pump use and water usage with and without small-area fixed effects.   % Taken together, the empirical evidence is supportive of the relatively strong assumptions necessary to identify booster pump preferences.


\section{Results}

Table~\ref{table:mainregs} includes results of the effect of price and pipe replacements.  Column (1) provides results on household water use finding an increase of 1.79 m3 per household per month after pipe replacement.  The estimate is statistically significant at the 1\% and economically large representing an 8\% increase in average consumption.  The estimated increase is smaller than the descriptive increase in Figure~\ref{figure:pipecons} likely because the regression approach controls for increasing trends in water use over the sample period.  

Column (1) also provides an estimated price sensitivity of 0.17, which is statistically significant at the 1\% level.  Given an average price of 21.5 PhP/m3, this estimate also implies a price elasticity of 0.17.  This elasticity estimate is on the lower end of similar studies in the developing world, which find elasticities ranging from 0.01 to 0.98.\footnote{See \cite{szabo2015value}, \cite{diakite2009proposal}, and \cite{strand2005water}.}  

Column (2) finds that pipe replacement projects are associated with a 20\% decrease in booster pump use, which is statistically significant at the 5\% level.  This decline is smaller than the descriptive decline in Table~\ref{table:descriptives} likely since the regression approach accounts for decreasing booster pump use over the sample period.

% While previous studies largely focus on cities in low-income countries (\cite{diakite2009proposal} ) or low-income neighborhoods within larger cities, this estimate  which may explain 

% dQ/Q / dP/P  =  (.17/22.14) /  (1/21.5)
% This estimate is in range of the 0.98 estimate from \cite{szabo2015value} as well as similar studies in the developing world that find elasticities between 0.01 and 0.81 (\cite{diakite2009proposal}, \cite{strand2005water}). 

\begin{table}[h!] 
\centering
\caption{Household Water and Booster Pump Use Estimates}\label{table:mainregs}
\vspace{-2mm} 
\begin{threeparttable}
\begin{tabular}{@{}l*{1}{CC}@{}}
\toprule
  & (1)       & (2)              \\
  & Water Use & Booster Pump Use \\
\midrule
After Pipe Replacement&        1.84\textsuperscript{a}&        1.79\textsuperscript{a}&       -0.20\textsuperscript{a}\\
                    &      (0.16)                   &      (0.15)                   &      (0.02)                   \\
Avg. Price (PhP)    &       -0.16\textsuperscript{b}&       -0.17\textsuperscript{a}&        0.01\textsuperscript{c}\\
                    &      (0.07)                   &      (0.07)                   &      (0.00)                   \\
Household Size      &        2.73\textsuperscript{a}&                               &        0.00                   \\
                    &      (0.03)                   &                               &      (0.00)                   \\
Employed Household Members&        0.34\textsuperscript{a}&                               &        0.00\textsuperscript{a}\\
                    &      (0.05)                   &                               &      (0.00)                   \\
High Skilled Employment&        1.85\textsuperscript{a}&                               &        0.07\textsuperscript{a}\\
                    &      (0.19)                   &                               &      (0.01)                   \\
Subdivided House/Duplex&       -2.40\textsuperscript{a}&                               &       -0.04\textsuperscript{a}\\
                    &      (0.15)                   &                               &      (0.01)                   \\
Freestanding House  &        1.37\textsuperscript{a}&                               &       -0.02\textsuperscript{a}\\
                    &      (0.14)                   &                               &      (0.01)                   \\
Ever High Price     &        1.38\textsuperscript{a}&                               &       -0.03                   \\
                    &      (0.41)                   &                               &      (0.02)                   \\
Change High to Low Price&        0.78\textsuperscript{c}&                               &       -0.01                   \\
                    &      (0.40)                   &                               &      (0.01)                   \\
Change Low to High Price&        1.09\textsuperscript{b}&                               &        0.03                   \\
                    &      (0.50)                   &                               &      (0.02)                   \\
Mean                &       22.14                   &       22.14                   &        0.16                   \\
Calendar Month FE   &  \checkmark                   &  \checkmark                   &  \checkmark                   \\
Small-Area FE       &  \checkmark                   &                               &  \checkmark                   \\
Household FE        &                               &  \checkmark                   &                               \\
$\text{R}^{2}$      &       0.223                   &       0.602                   &       0.344                   \\
N                   &   2,378,326                   &   2,378,326                   &      49,110                   \\
Dataset             &Billing Panel                   &Billing Panel                   &Household Survey                   \\

\bottomrule
\end{tabular}
\begin{tablenotes}
\footnotesize
\item Weighting, discussion of different samples, clustering, controls (especially rate classes).  This table predicts usage per household with pipe replacement and price with different fixed effects.  
\end{tablenotes}
\end{threeparttable}
\end{table}




\begin{table}[h!] 
\centering
\caption{Change in Consumer Surplus}\label{table:mainregs}
\vspace{-2mm}
\begin{threeparttable}
\begin{tabular}{@{}l*{1}{CCC}@{}}
\toprule
            & Service Quality              & Booster Pump Use & Total             \\
\midrule
Expression  & $\frac{w^{*}}{\alpha} \frac{dQ}{dR}$ &   $- F \frac{dB^{*}}{dR}$  & $\frac{dV}{dR}$ \\[1em]
Estimates   &   249.7 & 97.1 & 346.8 \\
            &  (109.6) & (16.8) & (103.7)  \\
% Evaluated Estimates \\
\bottomrule
\end{tabular}
\begin{tablenotes}
\footnotesize
\item Weighting, discussion of different samples, clustering, controls (especially rate classes).  This table predicts usage per household with pipe replacement and price with different fixed effects.  
\end{tablenotes}
\end{threeparttable}
\end{table}







Fixed costs:

On average, pipe replacement projects replace 25.3 km of pipes at a cost of 166 million PhP.  The average cost per pipe length is 9 million PhP per kilometer.  Since around 0.73 km of pipes are replaced per small area, total project costs per small area are around 6.6 million PhP.  With 275 connections per small area, project costs average 24,000 PhP per connection.   

6.6 million PhP/(275*1.41)


New Revenues:

Residential connections are billed 68 PhP more per month while commercial connections are billed 145 PhP more per month.  Given that 94\% of connections are residential and 6\% of commercial, the total billing increase is around 73 PhP per month per connection.  

Decrease in Marginal Costs:

After pipe-replacement, usage per connection decreases by 21 m3 per month.  Given marginal costs of 5 PhP per m3, then this decrease implies a cost reduction of 105 PhP per month.  


((73 )*12)/24000
((400)*12)/24000
((73 + 105 + 80)*12)/24000



\begin{table}[h!] 
\centering
\caption{Total Water Supplied and Billed Estimates}\label{table:nrwregs}
\vspace{-2mm}
\begin{threeparttable}
\begin{tabular}{@{}l*{1}{CCCCC}@{}}
\toprule
  & (1)   & (2)   & (3) \\
  & Volume Billed  & Volume Supplied & \% Non-Revenue Water \\
\midrule
After Pipe Replacement&     -187.55\textsuperscript{a}&      -37.51\textsuperscript{a}&       -1.48                   &       -0.31\textsuperscript{a}\\
                    &     (51.50)                   &     (10.30)                   &      (5.22)                   &      (0.06)                   \\
Mean                &      234.25                   &       46.85                   &       71.08                   &        0.28                   \\
$\text{R}^{2}$      &       0.880                   &       0.880                   &       0.947                   &       0.589                   \\
N                   &      63,546                   &      63,546                   &      64,842                   &      63,546                   \\

\bottomrule
\end{tabular}
\begin{tablenotes}
\footnotesize
\item Say it includes both fixed effects (Describe fixed effects...) .  Weighting, discussion of different samples, clustering, controls (especially rate classes).  This table predicts usage per household with pipe replacement and price with different fixed effects.   \regtext 45 PhP = 1 USD \,\,
\end{tablenotes}
\end{threeparttable}
\end{table}






\begin{table}[h!] 
\centering
\caption{Usage per Connection Regression Estimates}\label{table:profitregs}
\vspace{-2mm}
\begin{threeparttable}
\begin{tabular}{@{}l*{1}{cccc}@{}}
\toprule
  & (1) & (2) & (3) & (4)  \\
  & Usage (m3) & Bill (PhP) & Usage (m3) & Bill (PhP) \\
\midrule
After Pipe Replacement&        2.48\textsuperscript{a}&       67.82\textsuperscript{a}&        3.21\textsuperscript{a}&      144.76\textsuperscript{a}\\
                    &      (0.19)                   &      (6.90)                   &      (0.33)                   &     (26.43)                   \\
Mean                &       28.09                   &      682.44                   &       49.98                   &    3,253.21                   \\
Connection Type     & Residential                   & Residential                   &  Commercial                   &  Commercial                   \\
$\text{R}^{2}$      &       0.621                   &       0.616                   &       0.683                   &       0.718                   \\
N                   &   2,378,326                   &   2,377,563                   &   4,557,094                   &   4,550,439                   \\

\bottomrule
\end{tabular}
\begin{tablenotes}
\footnotesize
\item Weighting, discussion of different samples, clustering, controls (especially rate classes).  This table predicts usage per household with pipe replacement and price with different fixed effects.   \regtext 45 PhP = 1 USD \,\,
\end{tablenotes}
\end{threeparttable}
\end{table}



%%% leave out booster pumps?
\section{Counterfactuals}

% We're focusing on a part of the natural monopoly where the costs are easy to measure (you replace pipes), but the benefits are less well understood.  Different from the standard Laffont and Tirole model; closer to the Spence model which is a real contribution!  (luckily high cost areas also have high benefit)

% Set aside all other aspects of regulation to focus on quality!  Usually, capital is thought of as an input to deliver a certain quantity of the product.  Here we're focusing on a particular 

The following counterfactuals identify the welfare effects of optimal pipe replacements under five different information environments.  The \textit{full-information} counterfactual provides an optimal benchmark where the regulator chooses when to replace pipes to maximize total welfare given full information on demand and costs.  In the \textit{regular-replacement} counterfactual, the regulator simply replaces all pipes at a fixed interval.  In the \textit{quality-standards} counterfactual, the regulator establishes a minimum water pressure threshold and replaces pipes when pressure falls below the threshold.  In the \textit{cost-recovery} counterfactual, the regulator schedules replacements to balance savings from lower pumping costs against upfront costs of the pipe replacements.  In the \textit{firm-level} counterfactual, the regulator lets the firm decide when to replace pipes under the assumption that the firm has full demand and cost information.

% - hold price fixed!
% - decide replacements independently for each area
% - 

% - Not permitted expenses, similar to a price-cap [that's fair] (under invest)
% - Rate of return (allow for overlapping loans...?) --> no, don't ; make the point that the firm will want to replace pipes as soon as loans come due because the projects generate more than enough revenue to meet reasonable rates of return.  

% - permitted expenses or not? hard to know..
% - rate of return when we don't know the production function; 
% - high power --> redistribute little but keep costs down
% - low power --> redistribute a lot but high costs
% --- where are we?

% --> low power: all the surplus goes to consumers, but decisions are based on limited information
% --> high power: firm can get the surplus, but decisions are based on detailed consumer information; provides a strong incentive for the firm to figure out exactly the benefits of the investments
%%% in practice, it's somewhere between
%%% note that with the high power incentive, the prices are high, and the wedges get smaller if the prices are lower!
%%% footnote explaining distinction with standard high versus low-powered incentives.

% Since these projects generate streams of revenues in excess of their costs, then rate-of-return regulation would likely lead firms to very frequently replace pipes.\footnote{We do not explicitly model rate of return regulation here because the constraint would depend on whether the company could take out multiple loans, and the cost of capital for those loans.}


The \textit{full-information} counterfactual simulates how often a regulator would optimally replace pipes in order to maximize welfare under full information.  For each section of the service area, the regulator maximizes present-discounted welfare indefinitely given by the following expression
% doublecheck the tau's and t's
\begin{align*}
max_{R_t\in\{0,1\}} \,\, &\sum_{\tau = t}^{\infty} (1+\delta)^{t-\tau} [CS(k_{t}) + V(k_{t})]  \\
V(k_{t}) &= W(k_{t}) - E(k_{t}) - L(k_{t})\\ 
k_{t+1} &=  (1 + k_t) \, (1-R_t \mathbbm{1} \{ k_{t}> l\}) \\
L(k_t) &= \frac{r(1+r)^{l} }{(1+r)^{l} -1 }\overline{L} \, \mathbbm{1} \{ k_{t}\leq l \}
\end{align*}
where $t$ indexes month, $\delta(>0)$ is the discount factor, and $k_t$ is the age of the pipes.  $CS(k_{t})$ is consumer surplus and $V(k_t)$ is net revenue, which depends on revenue from water sales, $W(k_{t})$, costs of water leakage, $E(k_{t})$, and payments on any loans to fund pipe replacements, $L(k_t)$.  Due to pipe deterioration, consumer surplus and water revenue are both assumed to be decreasing in pipe age while water leakage is assumed to increase with pipe page.  $L(k_{t})$ is the payment on the loan needed to fund a pipe replacement.  Each month, the regulator chooses whether to replace pipes ($R_t = 1$) or not ($R_t = 0$).  Replacing pipes sets the pipe age, $k_t$, back to zero and is only possible if the regulator has paid off any debt from a previous pipe replacement where $l$ indexes the duration of the debt.  The regulator is assumed to fund a pipe replacement with a loan of total size, $\overline{L}$, at monthly interest rate, $r$, over duration, $l$.  The expression for $L(k_t)$ calculates the size of the monthly loan payments.  In this framework, the optimal schedule of pipe replacements balances net revenue against consumer surplus from new pipes.

In this framework, net revenues for a section of the service area can be positive, negative, or zero.  In practice, the regulator is likely to stagger pipe replacements across different sections of the service area so that total net revenues across all service areas are close to zero.  However, in some periods, total net revenues may still be negative or positive.  In these cases, the model implicitly assumes that the regulator adjusts fixed monthly service charges 

is able to maintain revenue neutrality by adjusting fixed monthly service charges paid by consumers.  



% Since this framework does not assume a revenue constraint at zero, 

% While in practice, regulators may adjust both marginal prices and fixed monthly service charges, the model focuses only on fixed monthly service charges  
% The model implicitly assumes that the regulator is able to implement fixed transfers to and from consumers without distorting their water consumption.  For example, the regulator may offer all consumers fixed rebates in months with budget surpluses and higher fixed water charges in months with budget deficits.  
% Allowing negative net revenues in the model provides an approximation for this scheduling optimization.

% While the model does not explicitly capture scheduling optimization, it 
This framework also holds the water tariff fixed.  



In the \textit{regular-replacement} counterfactual, the regulator is uninformed about consumer surplus and revenues.  Instead, the regulator only knows the age of the pipes and chooses to replace pipes as soon as they reach a certain age, $\overline{K}$.  In the \textit{quality-standards} counterfactual, the regulator observes not only the age of the pipes, but also the water pressure reaching consumers, $P(k_t)$.  The regulator then chooses to replace pipes once the water pressure falls below a certain threshold, $\overline{P}$.  In the \textit{cost-recovery} counterfactual, the regulator does not observe water pressure, but instead observes the part of producer surplus attributable to leakage costs, $E(k_t)$.  The regulator then replaces pipes to minimize leakage costs and pipe replacement costs.  



The marginal price is enough to cover the other fixed costs, and the firm gets to keep the leftovers!


In the \textit{firm-level} counterfactual, the regulated firm is assumed to be fully aware of producer surplus, $PS(k)$, and the regulator allows the firm to maximize producer surplus, $PS(k) - L(k_{t})$.

Total welfare versus producer/consumer divide?


% Mermelstein et al. (2012) for dynamic merger analysis
%%% in some months, it may be negative..

In the empirical set-up, ...


Discuss the empirical set up!  Discuss the limitations in the empirical set-up. Roughly correspond to natural divisions of the service area.

Consumer surplus and producer surplus are assumed to linearly decline in pipe age, $k_t$.  This assumption is consistent with civil engineering research documenting a roughly linear relationship between pipe failure and age.\footnote{See \cite{ward2017deterioration}, \cite{kleiner2001comprehensive}, and \cite{aydogdu2015estimation}.}  

Since the majority of pipes are replaced 

Appendix on decom.




Results are discussed below in Table [x].

 then the regulator maximization problem has an interior solution where the regulator choses whether to .  In practice, simulations solve the maximization problem recursively using 

% This approach abstracts away from many other features (price, independent decisions, maintenance, expectations about the future)

% where $t$ indexes month, $\delta>0$ is the discount factor, and $k_t$ is the age of the pipes.  $CS(k_{t})$ is consumer surplus and $PS(k_{t})$ is producer surplus, which are both assumed to be decreasing in pipe age as pipes deteriorate.  $L(k_{t})$ is the payment on the loans needed to fund upfront pipe replacements.  Each month, the regulator chooses whether to replace pipes ($R_t = 1$) or not ($R_t = 0$).  Replacing pipes sets the pipe age, $k_t$, back to zero and is only possible if the regulator has paid off any debt from a previous pipe replacement where $l$ indexes the duration of the debt.  The regulator is assumed to fund pipe replacements with a loan of total size, $\overline{L}$, at monthly interest rate, $r$, over duration, $l$.  The expression for $L(k_t)$ calculates the monthly loan payment for the duration of the loan.  In this framework, the optimal schedule of pipe replacements balances the (decreasing) consumer and producer surplus from new pipes against the upfront cost of replacing pipes.  

% In the \textit{regular-replacement} counterfactual, the regulator is uninformed about consumer and producer surplus.  Instead, the regulator only knows the age of the pipes and chooses to replace pipes as soon as they reach a certain age, $\overline{K}$.  In the \textit{quality-standards} counterfactual, the regulator observes not only the age of the pipes, but also the water pressure reaching consumers, $P(k_t)$.  The regulator then chooses to replace pipes once the water pressure falls below a certain threshold, $\overline{P}$.  In the \textit{cost-recovery} counterfactual, the regulator does not observe water pressure, but instead observes the part of producer surplus attributable to leakage costs, $E(k_t)$.  The regulator then replaces pipes to minimize leakage costs and pipe replacement costs.  

% In the \textit{firm-level} counterfactual, the regulated firm is assumed to be fully aware of producer surplus, $PS(k)$, and the regulator allows the firm to maximize producer surplus, $PS(k) - L(k_{t})$.

% Total welfare versus producer/consumer divide?


% k_{t+1} &= \begin{cases}
% k_t + 1 &\text{ if } R_t=0 \\
% 0 &\text{ if } R_t=1 \text{ and } k_{t}> l
% \end{cases}\\
% \begin{align*}
% L(k_t) = \begin{cases}
% \overline{L}\frac{r(1+r)^{l} }{(1+r)^{l} -1 } &\text{ if } k_{t}\leq l \\
% 0 &\text{ if } k_{t}> l
% \end{cases}
% \end{align*}
% d = (((1+r)^(12*loan_term))-1)/(r*(1+r)^(12*loan_term));
% F  = total_loan/d;




The fourth counterfactual imagines an uninformed regulator and a fully informed firm where the firm decides when to replace pipes.  The regulator does not adjust prices to compensate the firm for pipe replacements, which leaves the firm to maximize its own surplus from replacing pipes.\footnote{Strong incentive to reduce costs: This may occur when the regulator can't credibly commit to price increases or when the firm has a strong incentive to inflate costs which the regulator cannot verify.}




according to advice from water engineers who measure costs of replacements as well as pumping cost savings.  The regulator then replaces pipes to optimize upfront replacement costs against marginal savings in pumping costs.

without knowing the replacement costs or how replacements would affect water demand.  However, the regulator is able to measure water pressure either through consumer surveys, consumer complaints, or pressure gauges.  The regulator then establishes a minimum pressure threshold and replaces pipes when the pressure drops below the threshold.


Section Welfare:
	- Welfare benefits with 10 yr 5\% loans (assume discount rates) or 50 yr 5\% loan?
	- Welfare benefits with cash
	- Are households willing to pay for pipe improvements?

	- How do welfare estimates compare to traditional estimates from health measures?

Robustness:
	- Use nearest pipe replacement definition (selection into meter identification)



% - Discusses use of booster pumps in Clean Water Act 2004:  https://www.manilatimes.net/2018/11/04/legal-advice/dearpao/use-of-booster-pumps/461829/  http://r12.emb.gov.ph/ra-9275-the-philippine-clean-water-act/
% - https://idl-bnc-idrc.dspacedirect.org/bitstream/handle/10625/14878/108376.pdf?sequence=5
% - illegal booster pumps :  Consequence of El Niño 1997-98 on Manila water resources: new perspectives in water shortage management
% - seizing booster pumps : https://timesofindia.indiatimes.com/city/navi-mumbai/booster-pumps-used-by-some-socs-leading-to-water-crunch/articleshow/68856763.cms , https://timesofindia.indiatimes.com/city/nagpur/215-booster-pumps-seized/articleshow/69804589.cms




 % If households were sensitive to price increases at specific consumption levels, then households would be expected to frequently consume just below these levels.  However, Appendix Figure plots the frequency 






% The welfare benefits of fixing water pipes depends on which dimensions of service quality are improved by pipe replacement.  Table~\ref{table:descriptives} compares average household survey responses before and after pipe replacement.

% Allow services that they couldn't before

% Before pipe replacement, households experience weak water pressure and low water quality.  30\% of households report generally having no water pressure from 6 pm to 12 am in the absence of an extra household pump to boost water pressure.  23\% of households report that their piped water generally has foreign bodies.

% To compensate for low piped water quality, many households invest in private water booster pumps, storage drums, water filters, and purchases of non-piped water.  

% Both pressure and water quality improve substantially following pipe replacement.  The shares of households reporting no water pressure, water discoloration, foreign bodies, and unusual tastes/smells all drop to below 6\%.  

% - Quality complements more consumption
% - Water quality implies that consume the same, just enjoy it wayyy more
% 	- maybe need to clean things longer
% 	- maybe clean things the same but get way more utility

% - Pressure implies that quantity is positively linked to quality!
% 	- has the same feature?
% 	- use the same water, just less time...
% 	- costly to push for that extra water..... from a time perspective
% 		- Natural to think of relieving the pressure as leading to greater consumption!




	% Table: 
	% 	(A) What does pipe fixing mean?
	% 		- Quality, Hours, Flow

	% 	(B) How do households change their investments?

	% 	(C) How do households use their water?

	% 	(D) Household demographics including sharing
	% 		* N 
	% 	(E) Household consumption/payments/disconnection/standard deviation! (from billing data N)
	% 		* N




%%%% PIPE AGE %%%  possibly lower quality as well
% http://www.mayniladwater.com.ph/news-article.php?id=656
% West Zone concessionaire Maynilad Water Services, Inc. (Maynilad) replaced a total of 196 kilometers of old pipes in 2016, bringing the total length of pipes replaced within its concession area to 1,709 kilometers—about the same distance from Manila to Hanoi, Vietnam—since the company’s re-privatization in 2007.
% Maynilad invested ₱1.85 billion for its pipe replacement projects in 2016 alone. The old pipelines rehabilitated last year were located mostly in Quezon City, Manila and Bacoor City, where “pipe age” ranged from 30 to 40 years.
% http://www.mayniladwater.com.ph/news-article.php?id=366
% Maynilad spends 77M to replace 24km pipes in Malabon
% July 03, 2013

% West Zone concessionaire Maynilad Water Services, Inc. (Maynilad) is replacing 23.6 kilometers of secondary and tertiary pipes in Malabon City to improve services to more than 12,800 households in the area. The water company is spending some P77 million to replace the 30-year old pipes by the 3rd quarter of 2013.

% % doesn't dramatically change substitution to other sources!

% Reference Hanoi for how households may care about water use
% 	- percentages

% --- Water pump depreciation!?

% Figure consumption graph for effect dynamics (appendix with all other effect dynamics)

% Figures no bunching; small changes over time
% 	- Cite a bunch of other papers here... novel identification....

% Figures price variation and selection into switchers...
% 	- examine robustness to alternative alpha definitions...


\section{Appendix}







% \begin{table}[h!] 
% \centering
% \caption{Booster Pump Use Regression Estimates}\label{table:boosterregs}
% \vspace{-2mm}
% \begin{threeparttable}
% \begin{tabular}{@{}l*{1}{ccc}@{}}
% \toprule
%   & (1) & (2) & (3)  \\
% \midrule
% After Pipe Replacement&      -0.101\textsuperscript{a}&      -0.101\textsuperscript{a}&      -0.091\textsuperscript{a}\\
                    &     (0.011)                   &     (0.011)                   &     (0.008)                   \\
Pipe Replacement Area&       0.212\textsuperscript{a}&       0.212\textsuperscript{a}&                               \\
                    &     (0.015)                   &     (0.015)                   &                               \\
Avg. Price (PhP)    &                               &       0.001                   &       0.000                   \\
                    &                               &     (0.002)                   &     (0.002)                   \\
Ever High Price     &                               &      -0.012                   &       0.003                   \\
                    &                               &     (0.011)                   &     (0.010)                   \\
Ever Change Price   &                               &      -0.015                   &      -0.012                   \\
                    &                               &     (0.010)                   &     (0.010)                   \\
Household Size      &                               &      -0.001                   &       0.000                   \\
                    &                               &     (0.001)                   &     (0.001)                   \\
Employed Household Members&                               &       0.012\textsuperscript{a}&       0.010\textsuperscript{a}\\
                    &                               &     (0.002)                   &     (0.001)                   \\
High Skilled Employment&                               &       0.086\textsuperscript{a}&       0.081\textsuperscript{a}\\
                    &                               &     (0.007)                   &     (0.006)                   \\
Mean                &        0.15                   &        0.15                   &        0.15                   \\
Calendar Month FE   &  \checkmark                   &  \checkmark                   &  \checkmark                   \\
Small-Area FE       &                               &                               &  \checkmark                   \\
$\text{R}^{2}$      &       0.069                   &       0.076                   &       0.291                   \\
N                   &   2,379,456                   &   2,379,456                   &   2,379,456                   \\

% \bottomrule
% \end{tabular}
% \begin{tablenotes}
% \footnotesize
% \item This table predicts household booster pump use with pipe replacement including demographic controls and small-area fixed effects.   \regtext 45 PhP = 1 USD \,\,
% \end{tablenotes}
% \end{threeparttable}
% \end{table}







\begin{figure}
\caption{Price Time-Series}\label{figure:pricetimeseries}
\begin{center}
\includegraphics[scale=1]{tables/price_series.pdf}
\end{center}
\end{figure}


\begin{figure}
\caption{Tariff Schedule}
\begin{center}
\includegraphics[scale=1]{tables/rs_prices.pdf}
\end{center}
\end{figure}

\begin{figure}
\begin{center}
\caption{Consumption Histogram}
\includegraphics[scale=1]{tables/consumption_histogram.pdf}
\end{center}
\end{figure}



\begin{table}[h!] 
\centering
\caption{Water and Booster Pump Use per Household Estimates}\label{table:mainregshet}
\vspace{-2mm}
\begin{threeparttable}
\begin{tabular}{@{}l*{1}{CCCC}@{}}
\toprule
  & (1)       & (2)  & (3) & (4)            \\
  & Water Use & Water Use  & Booster Pump Use & Booster Pump Use \\
\midrule
Post                &        1.57\textsuperscript{a}&        1.69\textsuperscript{b}&       -0.21\textsuperscript{a}&       -0.21\textsuperscript{b}\\
                    &      (0.28)                   &      (0.80)                   &      (0.02)                   &      (0.09)                   \\
Avg. Price (PhP)    &       -0.15\textsuperscript{a}&       -0.11                   &        0.00                   &        0.00                   \\
                    &      (0.05)                   &      (0.16)                   &      (0.00)                   &      (0.00)                   \\
Post $\times$ Household Size&       -0.00                   &                               &       -0.00                   &                               \\
                    &      (0.05)                   &                               &      (0.00)                   &                               \\
Post $\times$ Employed Household Members&       -0.15\textsuperscript{c}&                               &        0.01                   &                               \\
                    &      (0.08)                   &                               &      (0.00)                   &                               \\
Post $\times$ High Skilled Employment&       -1.10\textsuperscript{a}&                               &       -0.01                   &                               \\
                    &      (0.28)                   &                               &      (0.02)                   &                               \\
Post $\times$ Subdivided House/Duplex&        0.59\textsuperscript{b}&                               &        0.01                   &                               \\
                    &      (0.25)                   &                               &      (0.01)                   &                               \\
Post $\times$ Freestanding House&        1.04\textsuperscript{a}&                               &        0.01                   &                               \\
                    &      (0.22)                   &                               &      (0.01)                   &                               \\
Post $\times$ Monthly Income (10,000 PhPs)&                               &       -0.38                   &                               &        0.05                   \\
                    &                               &      (0.43)                   &                               &      (0.05)                   \\
Household Size      &        1.68\textsuperscript{a}&                               &        0.00                   &                               \\
                    &      (0.09)                   &                               &      (0.00)                   &                               \\
Employed Household Members&        0.20                   &                               &        0.00\textsuperscript{b}&                               \\
                    &      (0.13)                   &                               &      (0.00)                   &                               \\
High Skilled Employment&        0.00                   &                               &        0.07\textsuperscript{a}&                               \\
                    &      (0.48)                   &                               &      (0.01)                   &                               \\
Subdivided House/Duplex&       -1.31\textsuperscript{a}&                               &       -0.05\textsuperscript{a}&                               \\
                    &      (0.40)                   &                               &      (0.01)                   &                               \\
Freestanding House  &        0.39                   &                               &       -0.02\textsuperscript{a}&                               \\
                    &      (0.33)                   &                               &      (0.01)                   &                               \\
Monthly Income (10,000 PhPs)&                               &        0.12                   &                               &        0.00                   \\
                    &                               &      (0.32)                   &                               &      (0.03)                   \\
Mean                &       19.83                   &       19.83                   &        0.16                   &        0.16                   \\
Household FE        &  \checkmark                   &  \checkmark                   &                               &                               \\
Small Area FE       &                               &                               &  \checkmark                   &  \checkmark                   \\
$\text{R}^{2}$      &       0.556                   &       0.582                   &       0.344                   &       0.320                   \\
N                   &   4,004,448                   &     501,075                   &      49,319                   &       6,041                   \\
Dataset             &Billing Panel                   &Billing Panel                   &Household Survey                   &Household Survey                   \\

\bottomrule
\end{tabular}
\begin{tablenotes}
\footnotesize
\item Weighting, discussion of different samples, clustering, controls (especially rate classes).  This table predicts usage per household with pipe replacement and price with different fixed effects.   \regtext 45 PhP = 1 USD \,\,
\end{tablenotes}
\end{threeparttable}
\end{table}



{
\small
\nocite{*}
\bibliographystyle{abbrvnat}
\setcitestyle{authoryear,open={((},close={))}}
\bibliography{library}
}



%\begin{center}
%  \input{mean_diff_leak1}
%\end{center}

%\begin{center}
%    \input{mean_diff_DC1}
%\end{center}



% \begin{table}[]
% \centering
% \caption{Source Choices}
% \label{my-label}
% \begin{tabular}{|l|c|c|c|}
% \hline
%                 & Fixed Price                  & Marginal Price                 & Tariff Kink Points      \\ \hline
% Individual (I)  & $F$                          & $P_k$                         & $\overline{w}_{k}$                         \\ \hline
% Shared   (S)    & $\frac{F}{2}$                & $P_k + P_{h}$                 & $\frac{\overline{w}_{k}}{2}$                            \\ \hline
% Vendor   (V)    & $0$                          & $P_{v}$                        & $0$                          \\ \hline
% \end{tabular}
% \end{table}
%\bibliography{lib}


\end{document}  